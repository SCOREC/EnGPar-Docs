% small.tex
\documentclass{beamer}
\usepackage[orientation=landscape, size = a0, scale=1.4]{beamerposter}
\usepackage[relative,overlay]{textpos}
\usetheme{Boadilla}
\usepackage{graphicx}
\usepackage{wrapfig}
%algorithms and pseudo code
\usepackage{algorithm}
\usepackage[noend]{algpseudocode}
\usepackage{numprint}
\usepackage{subcaption}
\usepackage{media9}
\usepackage{bibentry}
\usepackage[justification=centering]{caption}
\usepackage{pagecolor}
\usepackage{background}
\usepackage{blindtext}
\usepackage{xcolor}
\usepackage{enumitem}
\newlist{subquestion}{enumerate}{1}
\setlist[subquestion,1]{label=(\alph*)}

%For example blocks
\setbeamercolor{block title example}{fg=red,bg=orange}
\setbeamercolor{block body example}{fg=cyan,bg=yellow}

\nobibliography*

%\setbeamertemplate{bibliography item}[text]
%\setbeamertemplate{author in head/foot}{\insertshortauthor}
%\setbeamertemplate{navigation symbols}{}

\newcommand{\lenitem}[2][.6\linewidth]{\parbox[t]{#1}{\strut #2\strut}}

\setbeamercolor{background canvas}{bg=cyan!50}

\begin{document}
\font\titlefont=cmr12 at 80pt
\title[Load Balancing of Unstructured Meshes]
      {\titlefont
        Dynamic Load Balancing of \\[0.3cm]Massively Parallel Unstructured Meshes
      }
      \font\authorfont=cmr12 at 40pt
      \author[G. Diamond]{\authorfont
        Gerrett Diamond, Cameron W. Smith, Mark S. Shephard\\
        Rensselaer Polytechnic Institute, USA}
      \date{}
      %----------- titlepage ----------------------------------------------%

      \begin{textblock}{15.8}(0.05,0.1)
        \titlepage
        \begin{textblock}{14}(1,-0.2)
          \begin{block}{}
            Turn this into a paragraph
            \begin{itemize}
            \item A partitioning tool to complement existing multi-level and geometric methods.
            \item Provides a diffusive load balancing algorithm for partition improvement and supports multi-criteria partitioning.
            \item Utilizes a specialized multigraph structure to represent relation based data, called the N-graph.
            \item Implemented to support efficient data parallel operations on accelerators and vector units in many core processors.
            \end{itemize}
          \end{block}
        \end{textblock}
      \end{textblock}

      \begin{textblock}{15.6}(0.2,4)
        \begin{textblock}{5}(0,0)
          \begin{block}{EnGPar: Diffusive Graph Partitioning}
            The N-graph is defined as the following:
            \begin{itemize}
            \item A set of vertices $V$ representing the atomic units of work.
            \item If using the traditional graph mode:
              \begin{itemize}
              \item $N$ sets of edges $E_0,...,E_{n-1}$ for each type of relation.
              \item Each edge connects two vertices $u,v \in V$.
              \end{itemize}
            \item If using the hypergraph mode:
              \begin{itemize}
              \item $N$ sets of hyperedges $H_0,...,H_{n-1}$ for each type of relation.
              \item $N$ sets of pins $P_0,...,P_{n-1}$ corresponding to each set of hyperedges.
              \item Each pin in $P_i$ connects a vertex, $v \in V$, to a hyperedge $h \in H_i$.
              \end{itemize}
            \end{itemize}
            Advantages
            \begin{itemize}
            \item Multiple edge types allow for partitioning of multiple criteria simultaneously.
            \item Easy to use with diffusive procedures - compliments other partition methods via incremental refinements.
            \end{itemize}
            Diffusive Approach
            \begin{itemize}
            \item Iteratively migrate small sets of vertices to (1) reduce the peak imbalance and (2) reduce the number of (hyper)edges that are cut between part boundaries.
            \end{itemize}
            Diffusive Iteration Stages
            \begin{itemize}
            \item Weight Computation - compute weights and exchange with peers.
            \item Targeting - determined how much weight each peer can accept.
            \item Selection - select vertices for migration.
            \item Migration - move elements to peers.
            \end{itemize}
          \end{block}
          \begin{block}{Discrete Event Simulations}

          \end{block}
        \end{textblock}
        \begin{textblock}{5}(5.3,0)
          \begin{block}{Finite Element Analysis}
            
          \end{block}
          \begin{block}{Mira Runs}

          \end{block}
        \end{textblock}
        \begin{textblock}{5}(10.6,0)
          \begin{block}{Overset Meshes}
            
          \end{block}
          \begin{block}{FUN3D}

          \end{block}
        \end{textblock}
      \end{textblock}

\end{document}

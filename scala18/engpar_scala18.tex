\documentclass[conference]{IEEEtran}
\IEEEoverridecommandlockouts
% The preceding line is only needed to identify funding in the first footnote. If that is unneeded, please comment it out.
\usepackage{cite}
\usepackage{amsmath,amssymb,amsfonts}
\usepackage{algorithmic}
\usepackage{graphicx}
\usepackage{textcomp}
\usepackage{xcolor}
\def\BibTeX{{\rm B\kern-.05em{\sc i\kern-.025em b}\kern-.08em
    T\kern-.1667em\lower.7ex\hbox{E}\kern-.125emX}}
\begin{document}

\title{Dynamic Load Balancing for Particle-in-Cell, CFD, and Discrete Event Simulations\\
\thanks{Identify applicable funding agency here. If none, delete this.}
}

\author{\IEEEauthorblockN{1\textsuperscript{st} Gerrett Diamond}
\IEEEauthorblockA{\textit{SCOREC} \\
\textit{Rensselaer Polytechnic Institute}\\
Troy, NY\\
diamog@rpi.edu}
\and
\IEEEauthorblockN{2\textsuperscript{nd} Cameron W. Smith}
\IEEEauthorblockA{\textit{SCOREC} \\
\textit{Rensselaer Polytechnic Institute}\\
Troy, NY\\
smithc11@rpi.edu}
\and
\IEEEauthorblockN{3\textsuperscript{rd} Mark S. Shephard}
\IEEEauthorblockA{\textit{SCOREC} \\
\textit{Rensselaer Polytechnic Institute}\\
Troy, NY\\
shephard@rpi.edu}
}

\maketitle

\begin{abstract}

\end{abstract}

\begin{IEEEkeywords}
component, formatting, style, styling, insert
\end{IEEEkeywords}

\section{Introduction}

\begin{itemize}
\item motivate dynamic load balancing
\item talk about how different applications have different partitioning needs
\item Mention that engpar is general to support a range of structures.
\end{itemize}

\section{EnGPar}

\begin{itemize}
\item Discuss the Ngraph in general graph terms with some figures
\item Discuss dynamic load balancing and the general diffusive steps
\end{itemize}

\section{Particle In Cell}

\begin{itemize}
\item Briefly discuss PIC/XGCM
\item Mention that the mesh partition is static.
\item Describe the approach for load balancing particles
\item Describe the weight diffusion algorithm.
\end{itemize}

\section{Computational Fluid Dynamics}

\begin{itemize}
\item Briefly discuss CFD (FUN3D/Phasta).
\item Discuss vertex-based partition vs. element-based partition.
\item Discuss how EnGPar/Ngraph can support both.
\item Discuss the current approach in FUN3D for partitioning and our ideas to improve it.
\item Mention the boundary stack and our approach to better represent it.
\end{itemize}

\section{Results}

\begin{itemize}
\item Compare parmetis graph partitions vs. Zoltan hypergraph partitions
\item Then compare how EnGPar can balance each.
\item Use FUN3D runs to compare how different metrics affect the runtime.
\end{itemize}


\section{Future Works}

\begin{itemize}
\item accelerators
\item Other ideas for load balancing vertex-based partitions?
\end{itemize}

{\color{red} Add bibliography stuff here}

\end{document}

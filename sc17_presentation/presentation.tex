% small.tex
\documentclass{beamer}
\usetheme{Boadilla}
\usepackage{graphicx}
\usepackage{wrapfig}
%algorithms and pseudo code
\usepackage{algorithm}
\usepackage[noend]{algpseudocode}
\usepackage{numprint}
\usepackage{subcaption}
\usepackage{media9}
\usepackage{bibentry}
\nobibliography*

\setbeamertemplate{bibliography item}[text]
\setbeamertemplate{author in head/foot}{\insertshortauthor}
\setbeamertemplate{navigation symbols}{}

\newcommand{\lenitem}[2][.6\linewidth]{\parbox[t]{#1}{\strut #2\strut}}
\newcommand{\outline}{
  \begin{frame}<beamer>
    \frametitle{Outline}
    \tableofcontents[currentsection]
  \end{frame}
}

\begin{document}

\title[Unstructured Mesh Workflows]
{
Dynamic Load Balancing of Massively Parallel Unstructured Meshes
}
\author{Gerrett Diamond, Cameron W. Smith, Mark S. Shephard}
%\email{diamog@rpi.edu}
%\author[smithc11@rpi.edu]{Cameron W. Smith\\
%  \smallskip
%  Committee:\\
%  Mark Shephard\\
%  Max Bloomfield, Christopher Carothers, Barbara Cutler, Onkar Sahni
%}

\institute[SCOREC]{
Scientific Computation Research Center \\
Rensselaer Polytechnic Institute
}

\date{November 13, 2017}

%----------- titlepage ----------------------------------------------%
\begin{frame}[plain]
  \titlepage
\end{frame}

%----------- outline ----------------------------------------------%
\begin{frame}
  \frametitle{Outline}
  \tableofcontents
\end{frame}

%----------------------------------------------------------------------%
%----------- Section --------------------------------------------------%
%----------------------------------------------------------------------%
\section{Partitioning and Load Balancing}
\begin{frame}
  \frametitle{Movtivation \& definitions}
  Many evolving distributed simulations have: \\
  \begin{itemize}
    \item Complex relational structures.
    \item Irregular forms of computational and communication costs.
    \item Evolving imbalance of work. %Define Imbalance
    \item Multiple levels of load.
  \end{itemize}
\end{frame}

\begin{frame}
  \frametitle{Common Methods for Partitioning}
  \begin{itemize}
  \item Multilevel Graph Methods %Discuss poor scaling
    \begin{itemize}
    \item ParMETIS
    \item Scotch
    \end{itemize}
  \item Geometric Methods %Require coordinates
    \begin{itemize}
    \item Recursive Coordinate Bisection (RCB)
    \item Recursive Inertial Bisection (RIB)
    \end{itemize}
  \item Diffusive Methods %Improve a partition efficiently
    \begin{itemize}
    \item ParMA
    \item EnGPar
    \end{itemize}
  \end{itemize}
\end{frame}

\section{EnGPar - a graph based diffusive load balancer}
\begin{frame}
  \frametitle{Software}
  EnGPar's source can be found at \url{https://github.com/SCOREC/EnGPar}.
  \begin{itemize}
  \item Written in C++ using MPI.
  \item Utilizes CMake for building.
  \item Depends on a SCOREC developed communication library, PCU.
  \item Future GPU work will use Kokkos.
  \end{itemize}
\end{frame}

\begin{frame}
  \frametitle{N-graph}
  %EnGPar utilizes an expanded multigraph structure called the N-graph.\\
  %THIS IS TOO MUCH TEXT PROBABLY...
  The N-graph has two modes of operation: traditional or hypergraph.\\
  \smallskip
  The N-graph is defined as the following:
  \begin{itemize}
  \item A set of vertices $V$ representing the atomic units of work.
  \item If using the traditional graph mode:
    \begin{itemize}
    \item $N$ sets of edges $E_0,...,E_{n-1}$ for each type of relation.
    \item Each edge connects two vertices $u,v \in V$.
    \end{itemize}
  \item If using the hypergraph mode:
    \begin{itemize}
    \item $N$ sets of hyperedges $H_0,...,H_{n-1}$ for each type of relation.
    \item $N$ sets of pins $P_0,...,P_{n-1}$ corresponding to each set of hyperedges.
    \item Each pin in $P_i$ connects a vertex, $v \in V$, to a hyperedge $h \in H_i$.
    \end{itemize}
  \end{itemize}
\end{frame}

\begin{frame}
  \frametitle{Mapping structures to the N-graph}
  %Before using EnGPar a simulation must first map its data to the N-graph
  To map to the N-graph simulations must:
  \begin{itemize}
  \item Define units of work as the vertices.
  \item Decide on the method of edges to use.
  \item Define the relations between the work as (hyper)edges.
  \end{itemize}
  
  %Figure showing the conversion from mesh to N-graph
  \begin{figure}
    \centering
    \includegraphics[width=.7\textwidth]{figures/exampleMesh2Graph.png}
  \end{figure}
\end{frame}

\begin{frame}
  \frametitle{Diffusive Partitioning}
  \begin{algorithm}[H]
    \caption{Diffusive Load Balancing Framework}
    \label{alg:engpar}
    \small
    \begin{algorithmic}[1]
      \Procedure{Balance}{$ngraph$,$dimensions$}
      \ForAll{$d \in dimensions$}
      \While{imbalance of $d >$ tolerance}
      \Call{RunStep}{$ngraph$,$d$}
      \If{Balancing Stagnates}
      \State break
      \EndIf
      \EndWhile
      \EndFor
      \EndProcedure

      \Procedure{RunStep}{$ngraph$,$d$}
      \State $sides = makeSides(ngraph)$
      \State $weights = makeWeights(ngraph,sides,d)$
      \State $targets = makeTargets(ngraph,sides,weights)$
      \State $queue = makeIterationQueue(ngraph)$
      \State $plan = select(ngraph,targets,queue)$
      \State $trim(ngraph,plan)$
      \State $ngraph.migrate(plan)$
      \EndProcedure
    \end{algorithmic}
  \end{algorithm}
\end{frame}

%More slides to describe each step with some detail
\begin{frame}
  \frametitle{Sides, Weights, Targets}
  
\end{frame}

\begin{frame}
  \frametitle{Iteration Queue}
  
\end{frame}

\begin{frame}
  \frametitle{Selection}
  
\end{frame}

\section{Comparison to ParMA}

\begin{frame}
  \frametitle{Experiments}
  %Describe the mesh, system run on, initial partition etc.
\end{frame}

%More slides to go over the results in the paper

\begin{frame}
  \frametitle{Future Work}
  Expanding the capabilities of EnGPar:
  \begin{itemize}
    \item Improve partitioning structures at very high part count.
  \item Supporting the diffusive algorithms on GPUs
  \end{itemize}
  Applying EnGPar to other applications:
  \begin{itemize}
  \item CODES - a discrete event simulation for communication on supercomputer networks.
  \item FUN3D - a computational fluid dynamic simulation...
  \item PHASTA - I have no idea...
  \end{itemize}
\end{frame}

\begin{frame}
  \frametitle{Citations}
\end{frame}
  
\end{document}

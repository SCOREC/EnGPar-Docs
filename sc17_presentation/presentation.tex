% small.tex
\documentclass{beamer}
\usetheme{Boadilla}
\usepackage{graphicx}
\usepackage{wrapfig}
%algorithms and pseudo code
\usepackage{algorithm}
\usepackage[noend]{algpseudocode}
\usepackage{numprint}
\usepackage{subcaption}
\usepackage{media9}
\usepackage{bibentry}
\nobibliography*

\setbeamertemplate{bibliography item}[text]
\setbeamertemplate{author in head/foot}{\insertshortauthor}
\setbeamertemplate{navigation symbols}{}

\newcommand{\lenitem}[2][.6\linewidth]{\parbox[t]{#1}{\strut #2\strut}}
\newcommand{\outline}{
  \begin{frame}<beamer>
    \frametitle{Outline}
    \tableofcontents[currentsection]
  \end{frame}
}

\begin{document}

\title[Unstructured Mesh Workflows]
{
Dynamic Load Balancing of Massively Parallel Unstructured Meshes
}
\author{Gerrett Diamond, Cameron W. Smith, Mark S. Shephard}
%\email{diamog@rpi.edu}
%\author[smithc11@rpi.edu]{Cameron W. Smith\\
%  \smallskip
%  Committee:\\
%  Mark Shephard\\
%  Max Bloomfield, Christopher Carothers, Barbara Cutler, Onkar Sahni
%}

\institute[SCOREC]{
Scientific Computation Research Center \\
Rensselaer Polytechnic Institute
}

\date{November 13, 2017}

%----------- titlepage ----------------------------------------------%
\begin{frame}[plain]
  \titlepage
\end{frame}

%----------- outline ----------------------------------------------%
\begin{frame}
  \frametitle{Outline}
  \tableofcontents
\end{frame}

%----------------------------------------------------------------------%
%----------- Section --------------------------------------------------%
%----------------------------------------------------------------------%
\section{Partitioning and Load Balancing}
\begin{frame}
  \frametitle{Movtivation \& definitions}
  Many evolving distributed simulations have: \\
  \begin{itemize}
    \item Complex relational structures.
    \item Irregular forms of computational and communication costs.
    \item Evolving imbalance of work. %Define Imbalance
    \item Multiple levels of load.
  \end{itemize}
\end{frame}

\begin{frame}
  \frametitle{Common Methods for Partitioning}
  \begin{itemize}
  \item Graph Methods %Discuss poor scaling
    \begin{itemize}
    \item METIS/ParMETIS
    \item Scotch
    \end{itemize}
  \item Geometric Methods %Require coordinates
    \begin{itemize}
    \item RIB
    \item RCB
    \end{itemize}
  \item Diffusive Methods %Improve a partition efficiently
    \begin{itemize}
    \item ParMA
    \item EnGPar
    \end{itemize}
  \end{itemize}
\end{frame}

\section{EnGPar - a graph based diffusive load balancer}
\begin{frame}
  \frametitle{Software}
  \begin{itemize}
  \item Code can be found at \url{https://github.com/SCOREC/EnGPar}.
  \item Written in C++ using MPI.
  \item Depends on a SCOREC developed communication library, PCU.
  \item Future GPU work will use Kokkos.
  \end{itemize}
\end{frame}

\begin{frame}
  \frametitle{N-graph}
  
\end{frame}

\begin{frame}
  \frametitle{Mapping structures to the N-graph}
\end{frame}

\begin{frame}
  \frametitle{Diffusive Partitioning}
\end{frame}

%More slides to desribe each step with some detail

\section{Comparison to ParMA}

\begin{frame}
  \frametitle{Experiments}
  %Describe the mesh, system run on, initial partition etc.
\end{frame}

%More slides to go over the results in the paper

\begin{frame}
  \frametitle{Future Work}
\end{frame}

\begin{frame}
  \frametitle{Citations}
\end{frame}
  
\end{document}

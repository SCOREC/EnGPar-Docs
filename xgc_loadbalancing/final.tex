\documentclass[a4paper]{article}

\usepackage{listings}
\usepackage{xcolor}
\usepackage{hyperref}
\usepackage{graphicx}
\usepackage{geometry}
\usepackage{algorithm}
\usepackage[noend]{algpseudocode}
\usepackage{amsmath}
\geometry{margin=1.5in}
\usepackage[english]{babel}
\usepackage{url}
\usepackage{titling}

\setlength{\droptitle}{-2em}

\title{EnGPar - Partitioning Strategy for XGC}

\author{Gerrett Diamond}

\date{\today}

\begin{document}

\maketitle

\section{Important Questions}
The following questions have assumed answers for the following formulation
and N-graph construction. If these assumptions are wrong then the below
statements are either not going to work or will require further work on
top of it.
\begin{enumerate}
\item Are the safe zones defined as an integer amount of the flux faces? (Assumed yes)
  {\color{red}
    \begin{itemize}
    \item No, but the idea below has been altered to allow general safe zones.
    \end{itemize}
  }
\item Do the particles have computational or communication dependencies on one another? (Assumed no)
  {\color{red}
    \begin{itemize}
    \item No
    \end{itemize}
  }
\end{enumerate}

\section{Formulation}
Here I define some notation to define XGC at a high level as I understand it. I tried to match the terminology that I have seen/heard used so far. \\

\begin{itemize}
\item Let $F = \{F_1,F_2, ..., F_n\}$ be the flux faces.
\item Let $G^a = \{G_1, G_2, ..., G_n\}$ be the groups such that
  \begin{itemize}
  \item $G_i$ is defined by $F_i\in F$.
  \item $G_i = \{F_{i-a},...,F_{i-1},F_i,F_{i+1},...,F_{i+a}\}$ 
  \end{itemize}
\item Let $S = \{S_1,S_2,...,S_n\}$ be the safe zones for each $G_i$.
  {\color{blue}
  \item Define $\bar{S_i}$ for each set of overlapping safe zones. Picture to be made later.
  }
\item Let $P$ be the set of particles.
\item For each particle $p\in P$, Let $T_p$ be the set of safe zones in $S$ that p resides in.   {\color{blue} $\forall p \in P, \exists ! i$ such that $T_p = \bar{S_i}$.
\item  Let $P_{\bar{S_i}}$ be the set of particles such that $\forall p \in P_{\bar{S_i}}, T_p = \bar{S_i}$
  }
\end{itemize}

\section{N-graph}
Here is an explanation of a way to build the N-graph given the information above.
\begin{enumerate}
  {\color{blue}
  \item For each process $t$, create a vertex for each $\bar{S_i}$ where $S_t \in \bar{S_i}$.
  }
\item The weight of the vertex is initially set to {\color{blue} $|P_{\bar{S_i}}|$. This may be zero if the process doesn't currently have any particles in $\bar{S_i}$ .
\item For each $\bar{S_i}$, add a hyperedge that will be shared across processes that have a vertex for $\bar{S_i}$.}
\item Add pins between each hyperedge and vertex that represent {\color{blue} $\bar{S_i}$.}
\end{enumerate}
With this formulation we can use some sort of weight diffusion that will
migrate portions of the weight of vertices across the hyperedges to other
processes. {\color{blue} See sections \ref{sec:select} and \ref{sec:migrate}
for details of this method.}
{
  \color{blue}
\section{Migration of particles}
After EnGPar has run a weight diffuser, the resulting partition will
describe the amount of particles that should exist on each process
for each set $P_{\bar{S_i}}$. XGC can select particles to satisfy the weight
distribution given by EnGPar. Any stategy can be employed to choose
these particles. One such example is: for each $P_{\bar{S_i}}$, we migrate particles
that are closer to the safe zone of the target part and prioritize sending
the further out particles. This strategy should reduce the number of times
a particle is migrated.
}
\section{Other Questions}
Here are some questions that may affect the above idea to a smaller degree.
\begin{enumerate}
\item Do particles have different computational load?
  {\color{red}
  \begin{itemize}
  \item Answer: No.
  \end{itemize}
  }
\item How hard would it be to ``compute'' $T_p$, $\bar{S_i}$, and $P_{\bar{S_i}}$?
\item Does the outside regions (near the X point) mess up any of the assumptions or formulations used here?
\end{enumerate}

\section{Other Pieces to Consider}
\begin{itemize}
\item Should we account for particles that have left the safe regions before or during EnGPar's load balancing?
  {\color{red}
    \begin{itemize}
    \item Particles that have left the safe zone must be migrated to another group before the next push operation can occur. For a given process, we can subtract the number of non-safe particles from the number of safe particles to determine the current load. Ideally, we will execute at most one migration event per push to minimize communication costs.
    \end{itemize}
  }
\item Is it possible to get a good partition using a standard graph partitioning method on this or a similar graph construction.
  {\color{red}
    \begin{itemize}
    \item Yes, but will likely be slower than a diffusive approach. Also doesn't directly support weight diffusion.
    \end{itemize}
  }

  \item This design kind of assumes one process per flux face, which as I understand is not the case. However, we should be able to extend this idea for more processes.
\end{itemize}

{ \color{blue}
\section{Weight Diffusive Selection}
\label{sec:select}
\section{Weight Migration}
\label{sec:migrate}
}
\end{document}

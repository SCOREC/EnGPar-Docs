\documentclass[a4paper]{article}

\usepackage{listings}
\usepackage{xcolor}
\usepackage{hyperref}
\usepackage{graphicx}
\usepackage{geometry}
\usepackage{algorithm}
\usepackage[noend]{algpseudocode}
\usepackage{amsmath}
\geometry{margin=1.5in}
\usepackage[english]{babel}
\usepackage{url}
\usepackage{titling}

\setlength{\droptitle}{-2em}

\title{EnGPar - Partitioning Strategy for XGC}

\author{Gerrett Diamond}

\date{\today}

\begin{document}

\maketitle

\section{Important Questions}
The following questions have assumed answers for the following formulation
and N-graph construction. If these assumptions are wrong then the below
statements are either not going to work or will require further work on
top of it.
\begin{enumerate}
\item Are the safe zones defined as an integer amount of the flux faces? (Assumed yes)
\item Do the particles have computational or communication dependencies on one another? (Assumed no)
\end{enumerate}

\section{Formulation}
Here I define some notation to define XGC at a high level as I understand it. I tried to match the terminology that I have seen/heard used so far. \\

\begin{itemize}
\item Let $F = \{F_1,F_2, ..., F_n\}$ be the flux faces.
\item Let $G^a = \{G_1, G_2, ..., G_n\}$ be the groups such that
  \begin{itemize}
  \item $G_i$ is defined by $F_i\in F$.
  \item $G_i = \{F_{i-a},...,F_{i-1},F_i,F_{i+1},...,F_{i+a}\}$ 
  \end{itemize}
\item Let $S^b = \{S_1,S_2,...,S_n\}$ be the safe zones for each $G_i$ such that
  \begin{itemize}
  \item $b \le a$
  \item $S_i = \{F_{i-b},...,F_{i-1},F_i,F_{i+1},...,F_{i+b}\}$ 
  \end{itemize}
\item Let $P$ be the set of particles.
\item For each particle $p\in P$, Let $T_p$ be the set of safe zones in $S_b$ that p resides in.
\item {\color{blue} Let $P_{T_p}$ be the set of particles that whose set of safe zones is $T_p$.}
\end{itemize}

Technically as long as we can determine $T_p$ easily and there are large amounts of these sets being equal both on part and across parts, then the below construction should be feasible.

\section{N-graph}
Here is an explanation of a way to build the N-graph given the information above.
\begin{enumerate}
\item For each process, create a vertex for each unique $T_p$ for which this process/group has a safe zone in $T_p$.
\item The weight of the vertex initially is equal to {\color{blue} $|P_{T_p}|$.}
\item For each unique $T_p$ set across all processes, add a hyperedge that will be shared across processes that have a vertex for $T_p$.
\item Add pins between each hyperedge and vertex that represent $T_p$.
\end{enumerate}
With this formulation we can use some sort of weight diffusion that will migrate portions of the weight of vertices across the hyperedges to other processes. 
{
  \color{blue}
\section{Migration of particles}
Assuming we keep track of $P_{T_p}$ outside of EnGPar
}
\section{Other Questions}
Here are some questions that may affect the above idea to a smaller degree.
\begin{enumerate}
\item Do particles have different computational load?
  \begin{itemize}
  \item Answer: No.
  \end{itemize}
\item How hard would it be to ``compute'' $T_p$ and the processors that have each unique $T_p$?
\item Does the outside regions (near the X point) mess up any of the assumptions or formulations used here?
\end{enumerate}

\section{Other Pieces to Consider}
\begin{itemize}
\item Should we account for particles that have left the safe regions before or during EnGPar's load balancing?
\item Is it possible to get a good partition using a standard graph partitioning method on this or a similar graph construction.
  \item This design kind of assumes one process per flux face, which as I understand is not the case. However, we should be able to extend this idea for more processes.
\end{itemize}
\end{document}

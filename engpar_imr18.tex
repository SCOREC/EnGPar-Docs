%%%%%%%%%%%%%%%%%%%% author.tex %%%%%%%%%%%%%%%%%%%%%%%%%%%%%%%%%%%
%
% sample root file for your "contribution" to a contributed volume
%
% Use this file as a template for your own input.
%
%%%%%%%%%%%%%%%% Springer %%%%%%%%%%%%%%%%%%%%%%%%%%%%%%%%%%


% RECOMMENDED %%%%%%%%%%%%%%%%%%%%%%%%%%%%%%%%%%%%%%%%%%%%%%%%%%%
\documentclass[graybox]{svmult}

% choose options for [] as required from the list
% in the Reference Guide

\usepackage{mathptmx}       % selects Times Roman as basic font
\usepackage{helvet}         % selects Helvetica as sans-serif font
\usepackage{courier}        % selects Courier as typewriter font
\usepackage{type1cm}        % activate if the above 3 fonts are
                            % not available on your system
%
\usepackage{makeidx}         % allows index generation
\usepackage{url}             % links
\usepackage{graphicx}        % standard LaTeX graphics tool
                             % when including figure files
\usepackage{multicol}        % used for the two-column index
\usepackage[bottom]{footmisc}% places footnotes at page bottom

% see the list of further useful packages
% in the Reference Guide

\makeindex             % used for the subject index
                       % please use the style svind.ist with
                       % your makeindex program

%%%%%%%%%%%%%%%%%%%%%%%%%%%%%%%%%%%%%%%%%%%%%%%%%%%%%%%%%%%%%%%%%%%%%%%%%%%%%%%%%%%%%%%%%

\begin{document}

\title*{Accelerated Load Balancing of Unstructured Meshes}
% Use \titlerunning{Short Title} for an abbreviated version of
% your contribution title if the original one is too long
\author{
Cameron W. Smith
and Lucas Davis
and Gerrett Diamond
and Mark S. Shephard
}
\institute{
  Cameron W. Smith \at Rensselaer Polytechnic Institute, Troy, NY \email{smithc11@rpi.edu}
  \and Lucas Davis \at Rensselaer Polytechnic Institute, Troy, NY \email{davisl3@rpi.edu}
  \and Gerrett Diamond \at Rensselaer Polytechnic Institute, Troy, NY \email{diamog@rpi.edu}
  \and Mark S. Shephard \at Rensselaer Polytechnic Institute, Troy, NY \email{shephard@rpi.edu}
}
\authorrunning{C.Smith et al.}

\maketitle

\abstract{
  gpus are in lots of big systems, we need to use them for load balancing
}

\section{Introduction} \label{sec:intro}

\begin{itemize}
  \item briefly motivate dynamic load balancing
  \item quantify how GPUs are providing the majority of computing performance (\# of systems with GPUs in top 10 systems of top500, graph500, HPCG)
  \item end with a sentence that says what engpar does (diffusion) and how we are
extending it to run on GPUs
\end{itemize}

\section{EnGPar Dynamic Load Balancing} \label{sec:engpar}

\begin{itemize}
  \item multi-graph, high-level diffusion algorithm (targeting, selection, migration)
  \item indicate that we will accelerate selection via BFS for distance computation and coloring for cavity selection
\end{itemize}

\section{Accelerating Distance Computation} \label{sec:dist}

provide results - serial vs kokkos w/cuda -
\url{https://bitbucket.org/c_smith/engpar_siampp18/src/master/}

\section{Accelerating Cavity Selection} \label{sec:select}

We need to implement gpu-parallel selection of cavities using graph edge (edges
form cavities!) coloring to avoid races on the writing/reading the 'needs to be
migrated' tag of graph vertices.
The current selection procedure traverses the each part's boundary based on
descending distance from the topological center.
We need to figure out how to use the distance information while allowing large
sets of cavities to be evaluated simultaneously.
Possible approaches:
\begin{enumerate}
  \item ignore use of distance - parts will be large for efficient gpu
    computation and there will be fewer aspect ratio issues
  \item bin the cavities based on distance - the number of bins should define
    sets that are as large as possible without degrading the quality of
    selection - we want to select the close cavities first
  \item ignore distance and process all cavities simultaneously then apply a
    filter based on distance to remove the cavities with the largest distance -
    gpus support parallel sort so this should work - kokkos has a sorting
    method:
    \url{https://github.com/kokkos/kokkos/issues/422} \label{sortingSelect}
\end{enumerate}

Of the three options listed above, it seems like \#\ref{sortingSelect} will
expose the most concurrency and require the fewest algorithm modifications.

\begin{itemize}
  \item provide results - serial vs kokkos w/cuda
  \item may want to use a simple graph instead of waiting to have a hypergraph coloring procedure; that could be in future work/conclusion
\end{itemize}

\section{Application Results} \label{sec:results}

\begin{itemize}
  \item Figure out a better title for this section
  \item flow control strong scaling case with 1.3B tets: \\
64Ki \url{https://zenodo.org/record/833519#.WztuxXWYV1M} \\
128Ki \url{https://zenodo.org/record/834946#.Wztu-HWYV1M} \\
256Ki \url{https://zenodo.org/record/835483#.WztvCXWYV1M} \\
512Ki \url{https://zenodo.org/record/835742#.WztvG3WYV1M}
  \item the number of elements per-process may be too small for nodes with large GPUs to run efficiently (data transfer may become more costly than running the selection on the CPU!) - especially as we approach 512Ki parts
  \item we will have to run the condense tool to create an 2Ki (640k elms/part),
    4Ki (320k), 8Ki (160k), 16Ki (80k), and 32Ki (40k) meshes -
    fun3d uses between 75M and 2.3M elements per GPU on summitdev (from siampp18
    presentation)
  \item run on ORNL titan or summit (if accessible)
  \item compare runtimes versus results from SC17 paper~\cite{engparSC17}
  \item use mesh vertex = graph vertex and mesh edge = graph edge for tests -
    will need to run MPI only engpar to establish the baseline performance
  \item plot the time spent in MPI only selection vs kokkos coloring selection
    vs part size - the comparison must start and end at equivalent points in the
    code - start before selection and end just before migration (or whatever the
    next stage) begins
  \item plot the breakdown of time spent in coloring selection vs part size -
    data transfer, color computation, selection (possibly broken into cavity
    selection and filtering for distance), tranferring the selection list
    back to the host
  \item I expect there will be some reduction in partition quality using the
    coloring based selection since we won't have as fine-grained control over
    the process as the MPI only procedure.  As long as the quality reduction is
    controlled and performance is better we should be OK.
\end{itemize}

\section{Closing Remarks} \label{sec:closing}
\begin{itemize}
  \item summarize results
  \item discuss hypergraph coloring and relation to distance-2 coloring for non-simplices (quads, hexs, prisms, pyramids, etc.)
\end{itemize}

\begin{acknowledgement}
This research was supported by the U.S. Department of Energy, Office of Science,
Office of Advanced Scientific Computing Research, under award DE-SC00066117
(FASTMath SciDAC Institute) and by the National Science Foundation under Grant
No. ACI 1533581, (SI2-SSE: Fast Dynamic Load Balancing Tools for Extreme Scale
Systems).

We gratefully acknowledge the use of the resources of the ABC and the XYZ.

Any opinions, findings, and conclusions or recommendations expressed in this
material are those of the author(s) and do not necessarily reflect the views
of the National Science Foundation.
\end{acknowledgement}

\bibliographystyle{acm}
\bibliography{scorec-refs/scorec-refs}
\end{document}
